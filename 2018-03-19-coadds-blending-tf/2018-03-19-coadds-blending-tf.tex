\documentclass{beamer}

\usepackage{beamerthemesplit}
\usepackage{verbatim}
\usepackage[normalem]{ulem}

\usepackage{xcolor}

\usepackage{hyperref}

\definecolor{gold}{rgb}{1.,0.84,0.}
\definecolor{brightred}{rgb}{1.,0.4,0.4}
\definecolor{mygray}{RGB}{200,200,200}
\definecolor{lightsteelblue}{RGB}{176,196,222}
\definecolor{lightskyblue}{RGB}{135,206,250}
\definecolor{cadetblue}{RGB}{95,158,160}

\usetheme{default}
\usecolortheme{mule}

\usefonttheme{serif}

%\DeclareGraphicsExtensions{.pdf,.png,.jpg}

\newcommand{\mcal}{\textsc{metacalibration}}
\newcommand{\Mcal}{\textsc{Metacalibration}}

\newcommand{\mcalR}{\mbox{\boldmath $R$}}
\newcommand{\mcalRscalar}{\mbox{$R$}}

\newcommand{\mcalRmean}{\mbox{\boldmath $\langle R \rangle$}}
\newcommand{\mcalRscalarmean}{\mbox{$\langle R \rangle$}}

\newcommand{\mcalRpsf}{$R^{p}$}
\newcommand{\mcalRpsfnoise}{$R^{p}_\eta$}
\newcommand{\mcalRo}{\mbox{\boldmath $R_o$}}
\newcommand{\mcalRnoise}{\mbox{\boldmath $R_\eta$}}

\newcommand{\mcalRmeanalpha}{\mbox{\boldmath $\langle R_\alpha \rangle$}}
\newcommand{\mcalRmeanbeta}{\mbox{\boldmath $\langle R_\beta \rangle$}}

\newcommand{\mcalRg}{\mbox{\boldmath $R_\gamma$}}
\newcommand{\mcalRS}{\mbox{\boldmath $R_S$}}
\newcommand{\mcalRgmean}{\mbox{\boldmath $\langle R_\gamma \rangle$}}
\newcommand{\mcalRSmean}{\mbox{\boldmath $\langle R_S \rangle$}}

\newcommand{\mcalRtwopt}{\mbox{\boldmath $R^{2pt}$}}
\newcommand{\mcalRtwoptmean}{\mbox{\boldmath $\langle R^{2pt} \rangle$}}


\newcommand{\mcalRmodel}{\mbox{\boldmath $R^{model}$}}
\newcommand{\mcalRnoisemodel}{\mbox{\boldmath $R^{model}_\eta$}}


\newcommand{\vecg}{\mbox{\boldmath $\gamma$}}
\newcommand{\vest}{\mbox{\boldmath $e$}}

\newcommand{\snr}{$S/N$}
\newcommand{\snT}{$(S/N)_{\textrm{size}}$}
%\newcommand{\snT}{$\left( \frac{S}{N}\right)_{\textrm{size}}$}
\newcommand{\snflux}{$(S/N)_{\textrm{flux}}$}
%\newcommand{\snflux}{$\left( \frac{S}{N}\right)_{\textrm{flux}}$}

\newcommand{\lensfit}{\texttt{LENSFIT}}
\newcommand{\numba}{\texttt{Numba}}
\newcommand{\python}{\texttt{Python}}
\newcommand{\ngmix}{\texttt{ngmix}}
\newcommand{\ngmixer}{\texttt{ngmixer}}
\newcommand{\shear}{{\bf g}}
\newcommand{\redmapper}{redMaPPer}
\newcommand{\est}{$e$}


\newcommand{\prelim}{{\bf{\it Preliminary}}}

\newcommand{\uberseg}{{\color{lightsteelblue} {\"u}berseg}}
\newcommand{\MOF}{{\color{brightred}MOF}}


\title{Using Coadds for Photometry and Shear}
\author{Erin Sheldon}
\institute{Brookhaven National Laboratory}

% http://texblog.net/latex-archive/plaintex/beamer-footline-frame-number/
% to add the page (frame ) number and not screw up the bottom line
% works for split themes?
\expandafter\def\expandafter\insertshorttitle\expandafter{%
      \insertshorttitle\hfill%
        \insertframenumber\,/\,\inserttotalframenumber}

% suppress navigation bar
\beamertemplatenavigationsymbolsempty
\setbeamertemplate{footline}{}

\begin{document}

\usebackgroundtemplate{%
    \includegraphics[height=\paperheight]{des0022-4831-four.jpg}
}
\frame
{
}
\setbeamertemplate{background canvas}[vertical shading][bottom=mgray,top=mblack]

\setbeamerfont*{itemize/enumerate body}{size=\Large}
\setbeamerfont*{itemize/enumerate subbody}{parent=itemize/enumerate body}
\setbeamerfont*{itemize/enumerate subsubbody}{parent=itemize/enumerate body}


\frame{\titlepage}




\frame
{
    \frametitle{Outline}

    \begin{itemize}

        \item Introduction to coadds: pros and cons
        \item Mitigating issues with coadds
            \begin{itemize}
                \item Kaiser coadds
                \item Coadd the PSF
                \item Per-object coadds to deal with PSF discontinuities
                \item Dealing with correlated noise
            \end{itemize}
        \item Simulation tests
            \begin{itemize}
                \item Noise tests
                \item Shear bias tests
            \end{itemize}
        \item Future Work

    \end{itemize}

}

\frame
{
    \frametitle{Coadds: pros}

    \begin{itemize}

        \item Large data compression:  factor of about 100 for LSST

        \item Corresponding speed up for analysis

        \item Simplifies analysis, don't need to do multi-epoch fitting (but do
            still need multi-band)

    \end{itemize}

}

\frame
{
    \frametitle{Coadds: cons}

    \begin{itemize}

        \item The variance of quantities derived from standard coadds is
            generally higher than a multi-epoch fitting appraoch

        \item The PSF, noise, etc. discontinuous in the coadd at the location
            of edges in original images

        \item The noise is correlated in standard coadds due to interpolation


    \end{itemize}

}

\frame
{
    \frametitle{Example of Increased Variance}
 
 
    \begin{itemize}
        \item Template flux, just fitting for an amplitude $A$
        \item Gaussian PSF and object
        \item Approximate analytic formula can be derived (Sheldon, Armstrong, Huff, et al. in prep)
            {\normalsize
                \begin{align}
                    \textrm{var}{\hat{A_c}} = 
                    \textrm{var}{\hat{A}}\left[ 1 + \frac{1}{N_{epoch}}  \left( \frac{S}{N} \right)^2 (1-R)^2 \left( \frac{\Delta \sigma_p}{\sigma_p} \right)^2 \right]
                \end{align}
             }

         \item where $g$ marks the object and $p$ the PSF
         \item $R = \sigma_g^2/(\sigma_p^2 + \sigma_g^2)$ is 1 for very small galaxies or stars

    \end{itemize}


}

\frame
{
    \frametitle{Comparison of formula with simulation}
 
    \begin{center}
        \includegraphics[width=\columnwidth]{template-var.png}
        \newline
        (Sheldon, Armstrong, Huff, et al. in prep)
    \end{center}


}

\frame
{
    \frametitle{Mitigating increased variance: Kaiser Coadds}

    \begin{itemize}

        \item Kaiser derived a more optimal coadd for which there should be no increase in variance

        \item However we use SWARP in DES, and LSST currently also uses standard coadds

        \item From now on all tests will be on standard coadds

    \end{itemize}

}



\frame
{
    \frametitle{Mitigating PSF issues: per object coadds}

    \begin{itemize}

        \item Overcome discontinuities by caodding the PSF reconstructions also (e.g. Annis et al. SDSS coadd)

        \item However if an object crosses an image boundary, the PSF will still not be correct

        \item Solution is per-object coadds: make a coadd for each object
            separately, only including epochs where the object does not hit an
            edge (I heard the idea first from Jim Bosch)


    \end{itemize}

}

\frame
{
    \frametitle{Mitigating correlated noise issues}

    \begin{itemize}

        \item One can always whiten the noise, but it can result in a significant increase in noise

        \item Kaiser claims no loss of information if done correctly (I don't  understand this yet)

        \item Alternatively: determine the covariance and use it in analyses
            \begin{itemize}
                \item Run noise images through the coadd process also
                \item Feed this image into \mcal
                \item Measure a covariance matrix and use it in BFD (moments only taken once, so this is feasible)
            \end{itemize}


    \end{itemize}

}


\end{document}
