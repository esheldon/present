\documentclass{beamer}

\usepackage{beamerthemesplit}
\usepackage{verbatim}
\usepackage[normalem]{ulem}

\usepackage{xcolor}

\usepackage{hyperref}

\definecolor{gold}{rgb}{1.,0.84,0.}
\definecolor{brightred}{rgb}{1.,0.4,0.4}
\definecolor{mygray}{RGB}{200,200,200}
\definecolor{lightsteelblue}{RGB}{176,196,222}
\definecolor{lightskyblue}{RGB}{135,206,250}
\definecolor{cadetblue}{RGB}{95,158,160}

\usetheme{default}
\usecolortheme{mule}

\usefonttheme{serif}

%\DeclareGraphicsExtensions{.pdf,.png,.jpg}

\newcommand{\mcal}{\textsc{metacalibration}}
\newcommand{\Mcal}{\textsc{Metacalibration}}

\expandafter\def\expandafter\insertshorttitle\expandafter{%
      \insertshorttitle\hfill%
        \insertframenumber\,/\,\inserttotalframenumber}

% suppress navigation bar
\beamertemplatenavigationsymbolsempty
\setbeamertemplate{footline}{}


\title{Weak Lensing Algorithms and Lessons Learned from Precursor Experiments}
\author{Erin Sheldon}
\institute{Brookhaven National Laboratory}
\date{Match 19, 2020}
%

\begin{document}
\setbeamertemplate{background canvas}[vertical shading][bottom=mgray,top=mblack]

\setbeamerfont*{itemize/enumerate body}{size=\Large}
\setbeamerfont*{itemize/enumerate subbody}{parent=itemize/enumerate body}
\setbeamerfont*{itemize/enumerate subsubbody}{parent=itemize/enumerate body}


%
\frame
{

    % \frametitle{Introduction}

    {\huge The charge from the organizing committee: }
    \newline

    \begin{quote}
    ``...to speak on algorithms and pipelines you have developed or
    used in other surveys, lessons learnt, particularly any unexpected
    challenges you have encountered, and to highlight what you would do
    differently or additionally with LSST in order to maximize scientific
    return.''
    \end{quote}


}

\frame
{

    \frametitle{Outline}

    %\setbeamerfont*{itemize/enumerate body}{size=\large}

    \begin{itemize}

        \item Lessons from precursor surveys
        \item What we in DESC need for weak lensing shear
        \item What we in DESC are doing for weak lensing shear

    \end{itemize}

}


\frame
{
    \frametitle{Precursor Survey Work}

    \begin{itemize}

        \item SDSS shear pipeline development and lensing analysis

        \item BOSS target selection framework

        \item DES shear pipeline development and lensing analysis

        \item What did I/we learn from all this, and how are we implementing
            that knowledge for LSST?

    \end{itemize}

}

\frame
{
    \frametitle{Lesson: Identify the Critical Problems that are Unsolved}

    \begin{itemize}

        \item There are unsolved problems which {\em must be solved} if we are to succeed.

        \item Dedicate as many resources as possible to solving these problems

            \begin{itemize}
                \item This will require some {\em convincing}
            \end{itemize}

        \item Spend as little time as possible on solved problems (more on this later)

    \end{itemize}

}

\frame
{
    \frametitle{Example from My Experience}

    \setbeamerfont*{itemize/enumerate body}{size=\large}

    \begin{itemize}

        \item DES was proposed in about 2003 with weak lensing as a primary
            probe.  We thought we were close to a solution.

        \item Weak lensing measurement is still not a fully solved problem in
            2020, although recently candidate algorithms have been introduced
            that work in principle.

        % \item There are hundreds of people in the project whose science depends
        %     directly on the weak lensing calibration.

        \item  Over the years there has been at most a few persons full time
            sustained effort spent on this problem within DES, at a given time.
            At times this effort has been {\em less than one}.

        \item The situation is very similar in LSST right now.

    \end{itemize}

}

\frame
{
    \frametitle{Lesson: Use Existing Solutions if they are Good Enough}

    \setbeamerfont*{itemize/enumerate body}{size=\large}

    \begin{itemize}

        \item These solutions might not fit perfectly, I may have to bend
            them out of shape or adapt to their idiosyncrasies

        \item In most cases it will save so much time and frustration that it
            is worth it

        \item There is often more than one solution available, especially in
            cases where the software industry has already solved the problem.

        \item There are unsolved problems which {\em must be solved} if we are
            to succeed.  We must put our effort towards those problems.

    \end{itemize}

}

\frame
{
    \frametitle{Lesson: Aggressively test algorithms with validation simulations}

    \setbeamerfont*{itemize/enumerate body}{size=\large}

    \begin{itemize}

        \item Build up simulations in such a way that I can toggle all the
            relevant features, independently if possible

        \item Turn on features one at a time until I a) find a problem which I
            will try to fix or b) I reach full planned simulation complexity
            and the algorithm provides acceptable accuracy.

        \item Each test must be much more precise than the expected precision
            of the measurements I plan to make on real data.  For example, I
            want to be confident there is no bias with at least 99.7\%
            confidence.  This is necessary but can be resource intensive.

    \end{itemize}

}

\frame
{
    \frametitle{Lesson: Use best software practices, e.g. Unit Tests}

    \begin{itemize}

        \item Validation tests are in addition to unit tests, which may also be
            partly simulation based

        \item Use extensive unit tests and continuous integration (e.g. circleci
            on github)

    \end{itemize}

}

\frame
{

    \frametitle{Lesson: Confirmation bias is even more relevant for validation
    simulations than for real data analysis }

    \begin{itemize}

        \item The process is by definition iterative, I find problems and fix
            them

        \item I have many opportunities to fool myself:  I tend to stop when I
            measure no bias and move on to the next test.

        \item Always quote least 99.7\% confidence regions.  I will run
            thousands of tests and 2 sigma fluctuations will be seen regularly.

    \end{itemize}

}

\frame
{

    \frametitle{Lesson: Calibration simulations are harder than good data analysis}

    \setbeamerfont*{itemize/enumerate body}{size=\small}

    \begin{itemize}

        \item Often with data analysis, I don't need to understand everything that
            I measured, or how it is measured.

            \begin{itemize}

                \item E.g., we don't do photometric calibration by simulating the
                    instrument.  Rather we calibrate to references in a regime
                    where the machine response is linear

            \end{itemize}


        \item But, if I want to use a sim for estimating a large correction
            factor, I must know what I am doing. I need to understand what I
            put into the sim and it must match the real world very well.

        \item I should put as much effort as possible into developing
            algorithms that require small corrections.

            \begin{itemize}
    
                \item If the correction is 1\% I will be less sensitive to the errors in
                    my simulations than if the correction is 10\%.

            \end{itemize}

    \end{itemize}

}


\end{document}
