\documentclass{beamer}

\usepackage{beamerthemesplit}
\usepackage{verbatim}
\usepackage[normalem]{ulem}

\usepackage{xcolor}

\usepackage{hyperref}

\definecolor{gold}{rgb}{1.,0.84,0.}
\definecolor{brightred}{rgb}{1.,0.4,0.4}
\definecolor{mygray}{RGB}{200,200,200}
\definecolor{lightsteelblue}{RGB}{176,196,222}
\definecolor{lightskyblue}{RGB}{135,206,250}
\definecolor{cadetblue}{RGB}{95,158,160}

\usetheme{default}
\usecolortheme{mule}

\usefonttheme{serif}

%\DeclareGraphicsExtensions{.pdf,.png,.jpg}

\newcommand{\mcal}{\textsc{metacalibration}}
\newcommand{\Mcal}{\textsc{Metacalibration}}

\newcommand{\mcalR}{\mbox{\boldmath $R$}}
\newcommand{\mcalRscalar}{\mbox{$R$}}

\newcommand{\mcalRmean}{\mbox{\boldmath $\langle R \rangle$}}
\newcommand{\mcalRscalarmean}{\mbox{$\langle R \rangle$}}

\newcommand{\mcalRpsf}{$R^{p}$}
\newcommand{\mcalRpsfnoise}{$R^{p}_\eta$}
\newcommand{\mcalRo}{\mbox{\boldmath $R_o$}}
\newcommand{\mcalRnoise}{\mbox{\boldmath $R_\eta$}}

\newcommand{\mcalRmeanalpha}{\mbox{\boldmath $\langle R_\alpha \rangle$}}
\newcommand{\mcalRmeanbeta}{\mbox{\boldmath $\langle R_\beta \rangle$}}

\newcommand{\mcalRg}{\mbox{\boldmath $R_\gamma$}}
\newcommand{\mcalRS}{\mbox{\boldmath $R_S$}}
\newcommand{\mcalRgmean}{\mbox{\boldmath $\langle R_\gamma \rangle$}}
\newcommand{\mcalRSmean}{\mbox{\boldmath $\langle R_S \rangle$}}

\newcommand{\mcalRtwopt}{\mbox{\boldmath $R^{2pt}$}}
\newcommand{\mcalRtwoptmean}{\mbox{\boldmath $\langle R^{2pt} \rangle$}}


\newcommand{\mcalRmodel}{\mbox{\boldmath $R^{model}$}}
\newcommand{\mcalRnoisemodel}{\mbox{\boldmath $R^{model}_\eta$}}


\newcommand{\vecg}{\mbox{\boldmath $\gamma$}}
\newcommand{\vest}{\mbox{\boldmath $e$}}

\newcommand{\snr}{$S/N$}
\newcommand{\snT}{$(S/N)_{\textrm{size}}$}
%\newcommand{\snT}{$\left( \frac{S}{N}\right)_{\textrm{size}}$}
\newcommand{\snflux}{$(S/N)_{\textrm{flux}}$}
%\newcommand{\snflux}{$\left( \frac{S}{N}\right)_{\textrm{flux}}$}

\newcommand{\lensfit}{\texttt{LENSFIT}}
\newcommand{\numba}{\texttt{Numba}}
\newcommand{\python}{\texttt{Python}}
\newcommand{\ngmix}{\texttt{ngmix}}
\newcommand{\ngmixer}{\texttt{ngmixer}}
\newcommand{\shear}{{\bf g}}
\newcommand{\redmapper}{redMaPPer}
\newcommand{\est}{$e$}


\newcommand{\prelim}{{\bf{\it Preliminary}}}

\newcommand{\uberseg}{{\color{lightsteelblue} {\"u}berseg}}
\newcommand{\MOF}{{\color{brightred}MOF}}


\title{\mcal\ update}
\author{Erin Sheldon}
\institute{Brookhaven National Laboratory}

% http://texblog.net/latex-archive/plaintex/beamer-footline-frame-number/
% to add the page (frame ) number and not screw up the bottom line
% works for split themes?
\expandafter\def\expandafter\insertshorttitle\expandafter{%
      \insertshorttitle\hfill%
        \insertframenumber\,/\,\inserttotalframenumber}

% suppress navigation bar
\beamertemplatenavigationsymbolsempty
\setbeamertemplate{footline}{}

\begin{document}


\frame{\titlepage}




\frame
{
    \frametitle{\mcal\ Status}

    \begin{itemize}

        \item A catalog exists on the full Y3 area: 
            {\color{gold} \texttt{y3v02-mcal-v01}}
            \begin{itemize}
                \item Shears measured using $riz$ images.
                \item PSFs from standard DESDM PSFEx 

                \item Deblending using MOF (multi-object fitting)
                    {\color{gold} \texttt{y3v02-mof-v01}}

                \item Astrometry using standard DESDM
                \item Database table \texttt{NSEVILLA.Y3V02\_MCAL\_001}
                \item Flat files also available
            \end{itemize}

    \end{itemize}

}

\frame
{
    \frametitle{\mcal\ PSF leakage}

    \begin{center}
        \includegraphics[width=0.9\columnwidth]{{y3v02-mcal-001-e-vs-epsf0-s2n-10.0-Tratio-0.50}.pdf}
    \end{center}
}

\frame
{
    \frametitle{\mcal+MOF Deblending tests}

    \begin{center}
        \includegraphics[width=\columnwidth]{{compare-mof-nbrdist-crop-neg}.png}
    \end{center}
}





\frame
{
    \frametitle{Planned Y3 Improvements: PSF}

    \begin{itemize}

        \item M. Jarvis has shown that we need improved PSF measurements for Y3

        \item Can now run \mcal\ using PSFs from PIFF (M. Jarvis et al.).  A
            small MOF+shear run was completed on a testbed using older PIFF
            outputs.

        \item Improved PIFF is available over a portion of the Y3 area.
            Will perform tests this week.

    \end{itemize}

}

\frame
{
    \frametitle{Planned Y3 Improvements: Astrometry}

    \begin{itemize}

        \item We also need improved astrometry for Y3 shear measurements.

        \item New astrometry available, including such features as tree rings
            (G. Bernstein)

        \item Should be straightforward to support this, but a little code
            needs to be written.

        \item Can read the astrometry ``on the fly'' in the \texttt{ngmixer}
            codebase, already did that for SV

        \item Can also ``insert'' the corrections into the MEDS files.   Just
            modify the jacobians and postage-stamp coordinates.

    \end{itemize}

}

\frame
{
    \frametitle{Planned Y3 Improvements: Simulations}

    \begin{itemize}

        \item A significant set of simulations is planned
            for testing shear recovery end-to-end (McCrann, et al.).

        \item Should be useful for \mcal, BFD and MCCL

        \item Plan to use these for validation, not calibration.  
            Should provide an improved test of deblending and
            shear recovery.

        \item Should think a bit how we want to put shears in, and how to
            analyze.  The current sim has high shear, up to 0.08, which will
            cause a bias in \mcal\ and BFD due to a breakdown of the weak shear
            approximation.

    \end{itemize}

}

\frame
{
    \frametitle{Planned Y3 Improvements: Balrog}

    \begin{itemize}

        \item Inserting objects into the single-epoch data, and
            repeating data reduction through to coadds, 
            deblending, photozs and shear recovery.

        \item How shall we validate the deblending and shear recovery
            using Balrog?

        \item What kind of shear pattern should we use?  There can be
            additive errors, so we probably should use non-constant
            shear and fit for $m, c$ etc.


    \end{itemize}

}


\frame
{
    \frametitle{Possible Improvements: Deblending with new PSFs}

    \begin{itemize}

        \item The deblending will also benefit from improved
            PSFs and astrometry.

        \item Re-running MOF may not be needed: The deblending involves
            subtracting light from neighbors, PSF and astrometry effects should
            be less important.

    \end{itemize}

}

\frame
{
    \frametitle{Possible Improvements: New Deblender}

    \begin{itemize}

        \item Two new deblenders are in development, see 
            the Deblending II session on Thursday

        \item Probably a Y5 goal, unless we find deblending is
            the limiting factor for our cosmology analysis.

    \end{itemize}

}



\end{document}
