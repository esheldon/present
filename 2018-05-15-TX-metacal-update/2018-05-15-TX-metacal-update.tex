\documentclass{beamer}

\usepackage{beamerthemesplit}
\usepackage{verbatim}
\usepackage[normalem]{ulem}

\usepackage{xcolor}

\usepackage{hyperref}

\definecolor{gold}{rgb}{1.,0.84,0.}
\definecolor{brightred}{rgb}{1.,0.4,0.4}
\definecolor{mygray}{RGB}{200,200,200}
\definecolor{lightsteelblue}{RGB}{176,196,222}
\definecolor{lightskyblue}{RGB}{135,206,250}
\definecolor{cadetblue}{RGB}{95,158,160}

\usetheme{default}
\usecolortheme{mule}

\usefonttheme{serif}

%\DeclareGraphicsExtensions{.pdf,.png,.jpg}

\newcommand{\mcal}{\textsc{metacalibration}}
\newcommand{\Mcal}{\textsc{Metacalibration}}

\newcommand{\mcalR}{\mbox{\boldmath $R$}}
\newcommand{\mcalRscalar}{\mbox{$R$}}

\newcommand{\mcalRmean}{\mbox{\boldmath $\langle R \rangle$}}
\newcommand{\mcalRscalarmean}{\mbox{$\langle R \rangle$}}

\newcommand{\mcalRpsf}{$R^{p}$}
\newcommand{\mcalRpsfnoise}{$R^{p}_\eta$}
\newcommand{\mcalRo}{\mbox{\boldmath $R_o$}}
\newcommand{\mcalRnoise}{\mbox{\boldmath $R_\eta$}}

\newcommand{\mcalRmeanalpha}{\mbox{\boldmath $\langle R_\alpha \rangle$}}
\newcommand{\mcalRmeanbeta}{\mbox{\boldmath $\langle R_\beta \rangle$}}

\newcommand{\mcalRg}{\mbox{\boldmath $R_\gamma$}}
\newcommand{\mcalRS}{\mbox{\boldmath $R_S$}}
\newcommand{\mcalRgmean}{\mbox{\boldmath $\langle R_\gamma \rangle$}}
\newcommand{\mcalRSmean}{\mbox{\boldmath $\langle R_S \rangle$}}

\newcommand{\mcalRtwopt}{\mbox{\boldmath $R^{2pt}$}}
\newcommand{\mcalRtwoptmean}{\mbox{\boldmath $\langle R^{2pt} \rangle$}}


\newcommand{\mcalRmodel}{\mbox{\boldmath $R^{model}$}}
\newcommand{\mcalRnoisemodel}{\mbox{\boldmath $R^{model}_\eta$}}


\newcommand{\vecg}{\mbox{\boldmath $\gamma$}}
\newcommand{\vest}{\mbox{\boldmath $e$}}

\newcommand{\snr}{$S/N$}
\newcommand{\snT}{$(S/N)_{\textrm{size}}$}
%\newcommand{\snT}{$\left( \frac{S}{N}\right)_{\textrm{size}}$}
\newcommand{\snflux}{$(S/N)_{\textrm{flux}}$}
%\newcommand{\snflux}{$\left( \frac{S}{N}\right)_{\textrm{flux}}$}

\newcommand{\lensfit}{\texttt{LENSFIT}}
\newcommand{\numba}{\texttt{Numba}}
\newcommand{\python}{\texttt{Python}}
\newcommand{\ngmix}{\texttt{ngmix}}
\newcommand{\ngmixer}{\texttt{ngmixer}}
\newcommand{\shear}{{\bf g}}
\newcommand{\redmapper}{redMaPPer}
\newcommand{\est}{$e$}


\newcommand{\prelim}{{\bf{\it Preliminary}}}

\newcommand{\uberseg}{{\color{lightsteelblue} {\"u}berseg}}
\newcommand{\MOF}{{\color{brightred}MOF}}


\title{\mcal\ update}
\author{Erin Sheldon}
\institute{Brookhaven National Laboratory}

% http://texblog.net/latex-archive/plaintex/beamer-footline-frame-number/
% to add the page (frame ) number and not screw up the bottom line
% works for split themes?
\expandafter\def\expandafter\insertshorttitle\expandafter{%
      \insertshorttitle\hfill%
        \insertframenumber\,/\,\inserttotalframenumber}

% suppress navigation bar
\beamertemplatenavigationsymbolsempty
\setbeamertemplate{footline}{}

\begin{document}


\frame{\titlepage}




\frame
{
    \frametitle{\mcal\ Status}

    \begin{itemize}

        \item A catalog exists on the full Y3 area: 
            {\color{gold} \texttt{y3v02-mcal-v01}}
            \begin{itemize}
                \item Shears measured using $riz$ images.
                \item PSFs from standard DESDM PSFEx 
                \item Deblending using MOF $griz$ run
                    {\color{gold} \texttt{y3v02-mof-v01}}
                \item Astrometry using standard DESDM
                \item Database table \texttt{NSEVILLA.Y3V02\_MCAL\_001}
                \item Flat files also available
            \end{itemize}

    \end{itemize}

}

\frame
{
    \frametitle{\mcal\ PSF leakage}

    \begin{center}
        \includegraphics[width=0.9\columnwidth]{{y3v02-mcal-001-e-vs-epsf0-s2n-10.0-Tratio-0.50}.pdf}
    \end{center}
}

\frame
{
    \frametitle{\mcal+MOF Deblending tests}

    \begin{center}
        \includegraphics[width=\columnwidth]{{compare-mof-nbrdist-crop-neg}.png}
    \end{center}
}





\frame
{
    \frametitle{Planned Y3 Improvements: PSF}

    \begin{itemize}

        \item We think PSF improvements are needed for Y3
        \item Can now run \mcal\ using PSFs from PIFF (M. Jarvis et al.).  
            A small run was completed on a testbed using older PIFF outputs.
        \item Improved PIFF is available over a portion of the Y3 area.
            Will perform tests this week.

    \end{itemize}

}

\frame
{
    \frametitle{Planned Y3 Improvements: Astrometry}

    \begin{itemize}

        \item We think improved astrometry is needed for Y3.
        \item New astrometry available, including such
            features as tree rings (G. Bernstein)
        \item Should be straightforward to support this,
            but code needs to be written.


    \end{itemize}

}

\frame
{
    \frametitle{Planned Y3 Improvements: Simulations}

    \begin{itemize}

        \item A significant set of simulations is planned
            for testing shear recovery end-to-end (McCrann, et al.).

        \item Plan to use these for validation, not calibration.  
            Should provide an improved test of deblending and
            shear recovery.

    \end{itemize}

}

\frame
{
    \frametitle{Planned Y3 Improvements: Balrog}

    \begin{itemize}

        \item Inserting objects into the single-epoch data, and
            repeating data reduction through to coadds, 
            deblending, photozs and shear recovery.

        \item Hope to validate deblending and shear recovery

    \end{itemize}

}


\frame
{
    \frametitle{Possible Improvements: Deblending}

    \begin{itemize}

        \item The deblending will also benefit from improved
            PSFs and astrometry.

        \item Because the MOF is a slow code, we do not plan
            to re-run MOF at this time.

        \item May not be needed: Because the deblending involves
            subtracting light from neighbors, PSF and astrometry
            effects should be less important.


    \end{itemize}

}


\end{document}
